Optimization problems involving uncertainties are common in a variety
of engineering disciplines such as transportation systems, manufacturing,
communication networks, healthcare and finance. The large number of input
variables and the lack of a system model prohibit a precise analytical
solution and a viable alternative is to employ simulation-based
optimization. The idea here is to simulate a few times the stochastic
system under consideration while updating the system parameters until a
good enough solution is obtained.

� � � � � � Formally, given only noise-corrupted measurements of an
objective function, we wish to find a parameter which minimises the
objective function. Iterative algorithms using statistical methods search
the feasible region to improve upon the candidate parameter. Stochastic
approximation algorithms are best suited, most studied and applied
algorithms for finding solutions when the feasible region is a
continuously valued set. One can use information on the gradient/Hessian
of the objective to aid the search process. However, due to lack of
knowledge of the noise distribution, one needs to estimate the
gradient/Hessian from noisy samples of the cost function obtained from
simulation. Simple gradient search schemes take many iterations to
converge to a local minimum and are heavily dependent on the choice of
step-sizes. Stochastic Newton methods, on the other hand, can counter the
ill-conditioning of the objective function as they incorporate
second-order information into the stochastic updates. Stochastic Newton
methods are often more accurate than simple gradient search schemes.

We propose enhancements to the Hessian estimation scheme used in two recently proposed stochastic Newton methods, based on the ideas of random directions stochastic approximation (2RDSA) \cite{prashanth2015rdsa} and simultaneous perturbation stochastic approximation (2SPSA-3\footnote{The 3 in the abbereviation of the algorithm is used to indicate that the algorithm in \cite{bhatnagar2015simultaneous} requires 3 function evaluations per iteration.}) \cite{bhatnagar2015simultaneous}, respectively. The proposed scheme, inspired by \cite{spall-jacobian}, reduces the error in the Hessian estimate by 
\begin{inparaenum}[\bfseries (i)]
	\item incorporating a zero-mean feedback term; and
	\item optimizing the step-sizes used in the Hessian recursion.
\end{inparaenum}
We prove that both 2RDSA and 2SPSA-3 with our Hessian improvement scheme converges asymptotically to the true Hessian.
 %Moreover, for the special of a quadratic objective, we provide a convergence rate result for 2RDSA-IH that is of the same order - in the number of iterations of Hessian recursion - as that of 2SPSA with improved Hessian estimation (2SPSA-IH) \cite{spall-jacobian}. 
The key advantage with 2RDSA and 2SPSA-3  is that they require only 75\% of the simulation cost per-iteration for 2SPSA with improved Hessian estimation (2SPSA-IH) \cite{spall-jacobian}.
Numerical experiments show that 2RDSA-IH outperforms both 2SPSA-IH and 2RDSA without the improved Hessian estimation scheme.